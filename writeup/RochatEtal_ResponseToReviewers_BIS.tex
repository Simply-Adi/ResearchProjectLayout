\documentclass[11pt]{article}

%% WRY has commented out some unused packages %%
%% If needed, activate these by uncommenting
\usepackage{geometry}                % See geometry.pdf to learn the layout options. There are lots.
%\geometry{letterpaper}                   % ... or a4paper or a5paper or ...
\geometry{a4paper,left=2.5cm,right=2.5cm,top=2.5cm,bottom=2.5cm}
%\geometry{landscape}                % Activate for rotated page geometry
%\usepackage[parfill]{parskip}    % Activate to begin paragraphs with an empty line rather than an indent

\usepackage{natbib}
\bibliographystyle{chicago}

\usepackage{color}

%for figures
%\usepackage{graphicx}

\usepackage{color}
\definecolor{mygreen}{RGB}{28,172,0} % color values Red, Green, Blue
\definecolor{mylilas}{RGB}{170,55,241}
%% for graphics this one is also OK:
\usepackage{epsfig}

%% AMS mathsymbols are enabled with
\usepackage{amssymb,amsmath}

%% more options in enumerate
\usepackage{enumerate}
%\usepackage{enumitem}

%% insert code
\usepackage{listings}

\usepackage[utf8]{inputenc}
\usepackage{multicol}
\setlength{\columnsep}{1cm}

\usepackage{hyperref}

\usepackage{changes}


\usepackage{fourier, heuristica}
\usepackage{array, booktabs}
\usepackage{graphicx}

\newcommand{\R}[1]{\textbf{R}\textbf{#1}}
\newcommand{\ie}{i.e.}

\newcommand{\sZ}{\mathsf{Z}}
\newcommand{\sE}{\mathsf{E}}
%\newcommand{\sA}{\mathsf{A}}

\newcommand{\bdp}{\begin{description}}
\newcommand{\edp}{\end{description}}

\title{{\normalfont Reply to reviewers of the manuscript \texttt{2016GL070261} }
       ``Seasonality of submesoscale
       dynamics \\in the Kuroshio Extension''}
\author{Cesar B. Rocha,
        Sarah T. Gille, Teresa K. Chereskin, \& Dimitris Menemenlis}
\date{}
\begin{document}

%\include{symbols}

\maketitle

\section*{Summary}

\section{Reviewer \#1}

{\it The authors have adequately addressed all the questions raised in my
previous review. I recommend the manuscript be accepted for publication in GRL.}

\bdp
  \item We thank the reviewer for constructive review.
\edp

\section{Reviewer \#2}

{\it I have read the revised manuscript and the authors have adequately addressed
     my concerns. Here I would like to point out things I noticed.
     Please ignore those if you don't like.}

\bdp
   \item We thank the reviewer for constructive review.
\edp

\begin{enumerate}

\item {\it Line 166. "super-inertial IGWs" I think IGW represents any oscillation whose
                frequency is between f and N. If so, super-inertial is not needed.}

\item {\it Line 174. "inertia gravity waves" should be IGWs.}

\bdp
   \item Changed.
\edp

\item {\it Line 187. I am not sure if PSI is important near surface and wonder if it
                is resolved in this model because of the vertical resolution.
                Authors could discuss about PSI in the model with Jenn Mackinnon there.}

\item {\it Line 196. Remove "?"}

\bdp
  \item A reference  was amiss --- the ``?'' was raised during the\, \LaTeX\, compilation. Fixed.
\edp

\item {\it In supplement, page x-3 first line. Remove "r"}

\bdp
  \item Removed.
\edp

\item {\it To separate IGWs and other submesoscale motion, Ertel vorticity could be
      used (e.g. Kunze 1993, JPO, 2567-2588). This paper is about deep ocean though.}

      \bdp
        \item Thanks. Better to use APV.
      \edp


\end{enumerate}

\section{Reviewer \#3}

{\it The manuscript provides model-based evidence of a seasonal cycle in submesoscale motions
in the western North Pacific (Kuroshio Extension). The manuscript summarizes the results of
a study in which two years of moderate-resolution (1/24th degree or 4.6 km) and one year of
high-resolution (1/48th degree or 2.3 km) model output were examined for evidence of
submesoscale activity. Measures of submesoscale activity include vorticity, divergence and
strain rate and the authors demonstrate that these quantities are enhanced in winter. Kinetic
energy and sea surface height (SSH) spectra also show seasonal trends. These results are
consistent with an existing suite of studies reporting seasonality to submesoscale flows. What
is relevant for the present study and what distinguishes it from previous studies is that the
authors report an increase in internal wave activity during summer and which projects onto
the SSH variance. Because of the relevance of this study to challenges facing the upcoming
Surface Water Ocean Topography (SWOT) mission—which aims at measuring submesoscale
flows from space from nearly-balanced flows—this study will be of interest to readers of
GRL.


This is the second submission of the manuscript. The errors in the previous manuscript have
been addressed. The authors provide a direct connection between inertia-gravity waves and
SSH variance (Figure 4 of their manuscript) and illustrate that these undergo seasonality.
They have also framed the manuscript in terms of SWOT, which helps the reader understand
the importance of the results. There is no explicit connection between divergence and inertia gravity
waves (IGWs), making the connection between the two segments of the manuscript
tentative. (Part 1: analysis of the gradient tensor. Part 2: spectral analysis of SSH.) That is,
other ageostrophic flows can contribute to this divergence. Nevertheless, the seasonality in the
internal wave field is evident, complementing the observed seasonality in vorticity,
divergence and strain rate. In summary, despite the potential for varying internal wave
activity to be specifically tied to the Kuroshio Extension region, the implication of seasonal
variation in internal wave activity is new, and I recommend publication pending a few
revisions listed below.}

\bdp
   \item We thank the reviewer for the thorough and constructive review.
\edp

\begin{enumerate}
 \item {\it I appreciate the changes made by the authors. I am happy to know that the suggestions in
the previous review helped. Incidentally, I also appreciated the response by the authors
regarding the resolution of the model and scales energized by baroclinic instability, citing
work by Callies et al. (2016) and Larichev and Held (1995). Lastly, I am grateful for the
authors’ acknowledgement toward the end of the manuscript. However, I would rather the
authors remove me from the acknowledgements since I am involved in the formal review
process. There is no problem with this and I appreciate the authors’ efforts to thank me.}

\bdp
   \item We have removed the referee's name from the acknowledgements.
\edp

\item {\it 2. I have been looking at the $\omega-k$ spectra in Figure 4. There is a lot of interesting
information in these data. In particular, I have found three characteristics of interest. See
attached figure for my definition of corresponding regions.}

    \subitem (a) {\it Region 1: If one interprets submesoscale but larger-period waves (e.g., $\omega < f_{32.5}$ \& $k >
10-2$ km$^{-1}$
) as associated with gradient wind- or geostrophically-balanced flows and/or
front-related flows (note: I am not certain of this interpretation1
), there is considerable
change in energy from April to October, reflective of frontogenetic activity being
enhanced in late winter/early spring. This is consistent with the authors’ depiction of
surface fields in Figure 1d and 1e.}

  \subitem (b) {\it Region 2: Another interesting feature is the lack of energy in the inertia-gravity wave
(IGW) band (i.e., between mode-1 at $f_{40}$ and mode-4 at $f_{25}$) in April but presence of
this energy in October. This is a nice result.}

\subitem (c) {\it Region 3: Finally, the last feature I note is the change in energy at $\omega$ = 2f32.5 and at
horizontal wavenumbers $k > 8 \times 10-3$ km$^{-1}$ (i.e., horizontal wavelengths $< 125$ km).
Although these are IGWs themselves and may result from increased stratification as
mentioned by the authors, there still does seem to be elevated energy at $2f$ during
October. As one does not expect internal waves at the M2 frequency to change with
season, this seasonal change in energy is most likely explained by wave-wave
interactions. I confess I don’t know much about this topic but there is the suggestion
of such waves generated as a result of near-inertial wave interaction with quasigeostrophic
flows (Wagner and Young 2016). Perhaps the authors could discuss this
with GW or BY since they are mentioned in the acknowledgements. Niwa and Hibiya
(1999) also document such motions generated by an atmospheric disturbance and
which Olbers (1983) suggests can propagate very long distances. (Is it possible that
typhoon-related depressions that cross over the Kuroshio during October but not in
April excite inertial motions of frequency $2f$ and that then propagate into the analysis
region? The authors need not answer this in their paper ... Food for thought.) In
summary, I think (1) it would be worth mentioning that the energy at $\omega$ = $2f$ is
seasonally varying, as the authors are documenting, (2) that this projects onto SSH
variance at small horizontal scales, and (3) will be a source of energy seen by the
SWOT altimeter. This may be a result specific to the Kuroshio region but is
nevertheless interesting.}

\item {\it I recommend the authors use a different colormap when plotting spectra in Figure so that the
reader can see changes in internal wave energy more easily. Or perhaps contour levels?}

\item {\it As a side note, Reviewer 2 made a comment in the previous draft about near-inertial waves
(NIWs). S/he hypothesized that they may be an important source of SSH variance. Given the
above conversation, this Reviewer may be at least partly correct. It seems, though, as you say
that the waves near f penetrate deep into the bottom, since there is virtually no SSH variance
here}

\item{\it In the previous submission, I suggested estimating the magnitude of the sea surface height
signature within a particular band as a function of season. You have done this and it is nice.
Table 1 is also helpful for placing these quantities into context.

One question I have though is that, given that the bands described above have rather
complicated shape in $\omega-k$ space, might you instead integrate the PSD over these contoured
regions to determine the SSH variance found within each band and corresponding to each
type of motion? It would be interesting, for example, to know what fraction of the SSH signal
could be directly attributed to (1) IGWs motions within the dispersion relation curves and for
$k > 10-2$ km$^{-1}$ (region 2) (2) motions near $\omega$ = 2$f_{32.5}$ \& $k$ > 10-2 km$^{-1}$ (region 3) and (3) the
other portion of the spectra (region 1), which may reflect balanced and nearly-balanced
submesoscale motions. While the authors may not anticipate understanding energy within
these bands completely, future scientists may cite these motions as contributing to an
appreciable amount of the submesoscale SSH signal. If the authors feel this is not necessary
to include within the manuscript, this could be placed within the supporting material since it
would be interesting to the SWOT community}

\item{\it  There are a few wording mistakes that the authors should address prior to resubmission. I
don’t list them all but here are a few:}

  \subitem (a) {\it Figure 4, caption: “For reference, the gray curvish line represents” ==> Reword as,
  “For reference, we provide the baseline requirement as specified by the SWOT
  science team”, and provide a formal reference to where the reader can find this.}

  \bdp
      \item \noindent Done.
  \edp

  \subitem (b) {\it   Line 168: “The Similarly” ==> “Similarly...''}

  \bdp
      \item \noindent Fixed.
  \edp

  \subitem (c) {\it Lines 31-35: The revisit coverage of the SWOT spacecraft is on the order of 11 days
  (https://swot.jpl.nasa.gov/files/swot/SRD$\_$021215.pdf). I don’t really see how the
  inability to decouple geostrophic and ageostrophic motions presents an opportunity.
  But that being said, I would eliminate hyphens from line 33. It seems very informal.
  The colons in the sentence following this are fine.}

  \bdp
    \item {\noindent We removed the dashes. There are many opportunities associated with observations
          of ageostrophic submesoscale SSH from SWOT. At very least, the projection of both geostrophic
          and ageostrophic flows onto SSH --- and the interest in separating those flows --- will boost research
          on interactions between those types of motion. We have changed that sentence to make this point clear. }
  \edp

\end{enumerate}

\bibliography{rocha_etal.bib}


\end{document}
