\documentclass[11pt]{article}

%% WRY has commented out some unused packages %%
%% If needed, activate these by uncommenting
\usepackage{geometry}                % See geometry.pdf to learn the layout options. There are lots.
%\geometry{letterpaper}                   % ... or a4paper or a5paper or ...
\geometry{a4paper,left=2.5cm,right=2.5cm,top=2.5cm,bottom=2.5cm}
%\geometry{landscape}                % Activate for rotated page geometry
%\usepackage[parfill]{parskip}    % Activate to begin paragraphs with an empty line rather than an indent

\usepackage{natbib}
\bibliographystyle{chicago}

\usepackage{color}

%for figures
%\usepackage{graphicx}

\usepackage{color}
\definecolor{mygreen}{RGB}{28,172,0} % color values Red, Green, Blue
\definecolor{mylilas}{RGB}{170,55,241}
%% for graphics this one is also OK:
\usepackage{epsfig}

%% AMS mathsymbols are enabled with
\usepackage{amssymb,amsmath}

%% more options in enumerate
\usepackage{enumerate}
%\usepackage{enumitem}

%% insert code
\usepackage{listings}

\usepackage[utf8]{inputenc}
\usepackage{multicol}
\setlength{\columnsep}{1cm}

\usepackage{hyperref}

\usepackage{fourier, heuristica}
\usepackage{array, booktabs}
\usepackage{graphicx}
\usepackage[x11names]{xcolor}
\usepackage{colortbl}
\usepackage{caption}
\DeclareCaptionFont{blue}{\color{LightSteelBlue3}}

\newcommand{\foo}{\color{LightSteelBlue3}\makebox[0pt]{\textbullet}\hskip-0.5pt\vrule width 1pt\hspace{\labelsep}}

\newcommand{\R}[1]{\textbf{R}\textbf{#1}}
\newcommand{\ie}{i.e.}

\newcommand{\sZ}{\mathsf{Z}}
\newcommand{\sE}{\mathsf{E}}
%\newcommand{\sA}{\mathsf{A}}

\newcommand{\bdp}{\begin{description}}
\newcommand{\edp}{\end{description}}

\title{{\normalfont Reply to reviewers of the manuscript \texttt{2016GL070261} }
       ``Seasonality of submesoscale
       dynamics \\in the Kuroshio Extension''}
\author{Cesar B. Rocha,
        Sarah T. Gille, Teresa K. Chereskin, \& Dimitris Menemenlis}
\date{}
\begin{document}

%\include{symbols}

\maketitle

\section*{Summary}

In this revised manuscript, we have addressed the reviewers'
comments. Fundamentally, we have included a new wavenumber-frequency
spectral analysis suggested by reviewer 3. The new results vindicate
our claim that inertia-gravity waves, particularly super-inertial waves,
account for most of the summertime SSH variance and surface KE at scales
between 10 and 100 km. We have also expanded the introduction and discussion
to place our study in the context of Surface Water \& Ocean Topography (SWOT)
satellite mission.

We thank the reviewers  for the speedy review, constructive
criticism, and thoughtful suggestions that have helped us improve our discussion and
forced us to provide a better context for our study. We hope our manuscript is
now suitable for publication and look forward to your reply.


\section{Reviewer \#1}

{\it This is a timely investigation into the submesoscale sea surface height (SSH) and
velocity variations in the Kuroshio Extension region using global high-resolution
OGCM simulations. The main conclusion of the study that the surface ocean submesoscale
turbulence and inertia-gravity waves modulate out-of-phase seasonally, is novel and
important. Overall, the manuscript is well presented and clearly written and I
recommend its publication in GRL after the following comments are addressed.}\\

We thank reviewer \#1 for the speedy review. Below we address the reviewer’s
comments and suggestions.


\begin{enumerate}


\item {\it In the bottom paragraph on P.10, the authors emphasized that "At scales larger
than 20km, the April and October spectra based on hourly velocity snapshots are
nearly indistinguishable from each other". This emphasis appears to be contradictory
to the statements in other places of the manuscript that the seasonality of subinertial
submesoscale flows is strong (e.g., those relating to Figure 1b-c and Figure 2).}\\

    \bdp
        We agree that the emphasis on the similarity of hourly spectra can appear
        contradictory. Those spectra (now Figure 4c
        and supporting information) are similar because both submesoscale sub-inertial
        and super-inertial flows undergo a strong seasonal cycle that is out of
        phase --- there is a phase cancellation and the spectra based on hourly
        data are indistinguishable within errorbars. In any event, we removed
        that sentence from the text. The new Figure 4 should convey
        the phase cancellation idea more effectively.
    \edp

\item {\it Does the mesoscale eddy variability in the Kuroshio Extension region of the model
have a similar energy level as those in the along-track altimeter data? This point
is worth mentioning as it impacts on the relative importance between the instability-induced
turbulence versus inertia-gravity waves.}

  \bdp
      The model smoothed fields have EKE levels similar to across-track (JASON II)
      geostrophic EKE and gridded geostrophic EKE. We included this statement with
      the area-averaged EKE (lines 78-80). Thanks for pointing this out.
  \edp

\item {\it P.4, line 2 from bottom: Missing a word after "early".}

  Fixed.

\item {\it P.6, line 5-7: Should October 15 be part of "late spring/early summer"?}

  We rewrote that sentence to avoid ambiguity.

\item {\it P.7, line 12: Change "KE spectra" to "KE spectrum"?}

Fixed.

\item {\it P.8, bottom: The black line in Figure 2c shows a non-zero divergence value.
      Shouldn't the gridded SSH data be non-divergent by definition?}

  \bdp
    The convergence of meridians (the $\beta$-effect) accounts for the
    small divergence depicted by the black line in Figure 2c. We now added
    a brief explanation to the caption of figure 2.
  \edp

\item {\it P.11, line 4-6: Figure 4a reveals that daily averaging not only reduces the
      kinetic energy level for scales smaller than 250km, but also those longer
      than 500km. What part of the large-scale flow signals are being removed?}

  \bdp
  The large-scale super-inertial flows are barotropic tides. The new figure 4
  supports this interpretation --- there is significant semi-diurnal lunar
  tidal SSH variance (and KE) at scales larger than 500 km.
  \edp

\end{enumerate}

%
% Reviewer 2
%

\section{Reviewer \#2}
{\it In this paper, the authors describe seasonality of submesoscale (10-100 km)
activities in Kuroshio Extension region by using outputs from their new high
resolution numerical simulations. Authors found that submesoscale turbulence
dominates in late winter and early spring because of mixed layer instabilities, and
inertial-internal waves in late summer and early fall because of internal tides and
stratification near the surface. This new simulation, which includes tides and a high
frequency wind forcing, and authors' attempts to analyze surface submesoscale
turbulence and near-inertial waves are potentially important to oceanographic
community. However, some discussions seem weak, possibly because it is very
difficult to separate sub- and super-inertial motions in the submeso dynamics as
authors mentioned.}\\

We thank reviewer \#2 for the speedy review. Below we address the reviewer’s
comments and suggestions.

\subsection{Major points}

\begin{enumerate}

  \item {\it First of all, generation mechanism of submesoscale inertial-gravity waves in summer
        and early fall is unclear to me. In the paper, authors attribute the mechanism to
        internal tides (X-12, section 6, last sentence) and emphasize using the tidal forcing in
        their model (X-5, section 2, 2nd paragraph). However, authors do not explain or show
        any interaction mechanism of internal tides (please at least add some reference). With
        the fixed frequencies of tidal constituents and existence of the critical latitude for the
        diurnal tides, I am not sure what mechanism can represent the surface near-inertial
        wave generation within the whole analyzed region by internal tides (with the
        horizontal scales of 10-100km). If authors compare Figures 2 and 4 with those
        estimated from the non-tidal forcing run in the Supplemental material, authors'
        conclusion could be strengthened.}\\

        \bdp
          The generation mechanism for internal
          waves involves changes in the wind (NIWs) and barotropic tides sloshing over
          topographic features --- we are not claiming that there exists a different
          generation mechanism
          in summer compared to winter; neither are we claiming that surface NIWs
          are generated by internal tides. What is different is the near-surface
          expression of IGWs, which is very sensitive to the near-surface stratification
          \citep[; lines 191-196]{dasaro1978}.\\

          To reference potential mechanisms of how energy can quickly be transferred to
          high wavenumbers by interaction with mesoscale-submesoscale turbulence and
           PSI,  we added new references to the discussion \citep{mackinnon_winters2005,
           ponte_klein2015, alford_etal2016}. \\

          The simulation without tides (ECCO2 Adjoint) has much coarser
          horizontal resolution and barely resolves the submesoscales; its grid
          spacing is 18 km, while its effective resolution larger than 60 km.\\
      \edp


  \item {\it I also wonder why authors ignore wind-induced near-inertial waves (X-13, the first
        line). It is well studied that wind-induced near-inertial motion in mixed layer in back
        ground flow fields disperse quickly and propagate into the ocean interior (e.g., Young
        and Ben-Jelloul 1997, J. Mar. Res., 55, 735-66). The distribution of those waves
        depends on the distribution of vorticity (or Laplacian of vorticity) of background
        flows (Klein and Llewellyn Smith 2001, J. Mar. Res., 59, 697-723; Klein et al. 2004, Q.
        J. R. Meteorol. Soc., 130, 1153-1166). Therefore, those wind-induced near-inertial
        waves could have the horizontal scales of 10-100km. Note that those studies are based
        on geostrophically balanced or quasigeostrophic turbulent field. However, authors
        may need to study interactions between submeso dynamics regime (Rossby number
        and Richardson number are close to 1) and near-inertial waves in winter and spring
        cases where I guess that non-linear terms may play a role. I don't think that authors
        need to explain all internal wave dynamics in this GRL paper. I would rather suggest
        to remove all discussions related to submesoscale inertial-gravity waves and focus on
        describing differences between October and April.}\\

        \bdp
            We are now citing the recent review on NIWs by \cite{alford_etal2016} that contains
            a section dedicated to mesoscale-turbulence-NIW interactions ---  \cite{alford_etal2016}
            cite all the
            aforementioned papers (lines 185-190). The new discussion together with Figure 4
            and Figure S7 should clear any ambiguity. Those figures unambiguously
            show that most of the IGWs with horizontal scales between 10-100 km
            are mostly super-inertial (not near-inertial); the refracted/dispersed  NIWs
            propagate into the interior and only weakly project onto SSH \citep{alford_etal2016}.
        \edp

  \item {\it Finally, I don't understand why authors define 10-100 km as a submesoscale and used
        this scale for the spatial filtering (X-7, section 3). Authors should discuss Figure 4 first
        and then give a reason why they define 10-100 km as a submesoscale and use this
        scale for the filtering. In X-11, section 5, author point out that a part of the dynamics
        around 250 km scale should be explained by near-inertial waves. Then, I wonder why
        authors decide to discuss internal waves in 10-100 km scale.}

        \bdp
            Following reviewer \#3's suggestion, we have added paragraphs about SWOT
            to the introduction and conclusion. This discussion adds context to our study
            and justifies the focus on the horizontal scales between 10-100 km. Our definition
            of submesoscales have pros and cons (see reply to reviewer \#3's
            minor comment \#1) but is consistent with the definition used by
            (at least part of) the SWOT community \citep{fu_ferrari2008}. Giving the
            definition, which we present in the first sentence of the introduction,
            then there is no ambiguity in using 100 km as a cut-off scale for the filter.
            We have chosen to present bulk statistics first, and then discuss details of
            different scales --- this makes more sense now with the new wavenumber-frequency
            analysis.
        \edp

\end{enumerate}

\subsection{Minor points}

\begin{enumerate}

  \item {\it X-4, section 1, 2nd paragraph and X-13, section 6 last: Authors claimed that they
        analyze the Kuroshio Extension region. However, their study region does not
        include between $~$140E and $~$155E where the Kuroshio Extension is strongest in
        Figure 1(a). Please explain why you exclude this region in the paper. Since those
        longitudes includes the Izu-Ogasawara ridge (internal tide generation) and
        formation region of Subtropical mode water (strong mixed layer front), I would
        expect that including these latitudes would strengthen authors conclusions. I
        would also like to point out that positions of KEO and KESS moorings in the
        supplemental material are outside of the study region.}\\

        \bdp
             The upper bound on
             the size of the domain is chosen for lateral homogeneity purposes,
             necessary for the spectral analysis. We decided to focus in
             this subdomain to avoid the strong lateral inhomogeneity and
             anisotropies associated
             with the Kuroshio upstream and near its separation from Japan's coast.
             We have considered larger and smaller domains and found insignificant
             quantitative differences.\\

             The supporting information explicitly states that both KEO and KESS
             mooring were outside our chosen domain. We analyzed data from those
             mooring simply to assess the skill of numerical model in representing
             high-frequency modes. As discussed in the supporting information, the
             comparison uses model output at the grid points closest
             to the mooring.
        \edp

  \item {\it X-4, section 1, 3rd paragraph: “submesoscale inertial-gravity waves”. I feel that this
        nomenclature is too sudden. It might be better to write “inertial-gravity waves
        with a horizontal scales of 10-100 km (hereafter submesoscale inertial-gravity
        waves)”?}\\

        \bdp
          We included the definition (line 41). thanks for pointing that out.
        \edp

  \item {\it X-9, section 4, 2nd paragraph: I wonder what we could learn from the sentence “In
          other words, even in April, when submesoscale turbulence prevails, only the
          daily-averaged fields are largely in geostrophic balance”. This suggests that the
          balanced and upward cascade motion (originated from the submesoscale
          instability) scenario in Sasaki et al., (2014, nature comm. 5) could be wrong
          because they use the daily averaged data (X-11, section 5, last of 1st paragraph)?
          Or it just means that the balanced regime is actually the submeso-dynamics
          regime?}\\

          \bdp
              This simply means that there is significant ageostrophic components,
              even in April. Of course, daily-averaging the fields filters most of
              ageostrophic. This does not imply that Sasaki et al's mechanism is
              incorrect; it does suggests, however, a more complicated picture, with
              likely small energy leak towards small scales. In any event, to avoid ambiguity, we
              have re-written that sentence and excluded repeated information.
          \edp

  \item {\it X-10, section 4, 2nd paragraph, the last sentence: Please add some reference.}\\
          We have re-written that sentence.
\end{enumerate}


\section{Reviewer \# 3}
        {\it This manuscript provides model-based evidence of a seasonal cycle in submesoscale motions
         in the western North Pacific (Kuroshio Extension). The manuscript summarizes the results of
         a study in which two years of moderate-resolution (1/24th degree or 4.6 km) and one year of
         high-resolution (1/48th degree or 2.3 km) model output were examined for evidence of
         submesoscale activity. Measures of submesoscale activity include vorticity, divergence and
         strain rate and the authors demonstrate that these quantities are enhanced in winter. Kinetic
         energy and sea surface height (SSH) spectra also show seasonal trends. These results are
         consistent with an existing suite of studies reporting seasonality to submesoscale flows.

         What is relevant for the present study and distinguishes it from previous studies is that the
         authors report an increase in internal wave activity during summer and which (they suggest)
         projects onto the SSH variance. Because of the relevance of this study to challenges facing
         the upcoming Surface Water Ocean Topography (SWOT) mission—which aims at measuring
         submesoscale flows from space from nearly-balanced flows—this study will be of interest to
         readers of GRL. The major error that should be addressed prior to publication is a more direct
         connection of divergence and increased SSH variance to inertia-gravity waves. At present,
         this connection is entirely speculative since any ageostrophic flows can give rise to
        divergence and SSH variance. Edits should be made to the text and figures, as well.}\\

        We thank  the reviewer for the speedy and thorough review. Below we address
                  reviewer's comments and suggestions.


\subsection{Major comments}

\begin{enumerate}

  \item {\it The authors must clearly demonstrate that increased inertia-gravity waves (IGWs) are
        responsible for the seasonal signal in horizontal divergence and SSH spectra at high
        wavenumbers. The authors argue that this is so but it is not demonstrated with sufficient
        clarity. It is true that a portion of this energy/variance will be attributed to internal waves
        modified by rotation but other flows give rise to divergence (and thus, SSH variance). Thus,
        the seasonal changes reported by the authors might be the result of seasonal changes in these
        other flows. Given that this is the main message of the paper, the authors must make this
        modification.}\\

        \bdp
            We have included wavenumber-frequency
            spectra of SSH variance (new figure 4) and KE (figure S7) with bounds for the
            dispersion relationship of free IGWs, as suggested by the reviewer. The new
            results unambiguously
            indicate that IGWs account for most of the super-inertial variability at
            scales between 10-100 km. Thanks for the excellent suggestion!
        \edp

  \item {\it There is very little introduction in this manuscript. The authors must address this
          shortcoming prior to publication. The obvious method of addressing this issue is to discuss
          the seasonality of IGWs in the context of SWOT and why this might matter. The introduction
          should, therefore, place your study in the greater context and answer the question,
          ``Why does this study matter?''}\\

          \bdp
              We agree that the planning of SWOT is perhaps the most immediate
              application of our results. We have added an introductory paragraph dedicated to SWOT (
              lines 27-35). We have also added a discussion about the
              implication for SWOT in section 6. Thanks for making us provide a better
              context for our study.
          \edp

  \item {\it There needs to be a discussion about the effective resolution of your model. The fact that
        the model has a resolution of dx = 2.3 km and 4.6 km does not mean it can resolve oceanic
        phenomena at this scale. Its effective horizontal resolution is closer to 8 times the grid
        resolution (Soufflet et al. 2013). This corresponds to 18 km and 35 km for LLC4320 and
        LLC2160 output, respectively. [A more exact value could be obtained by estimating zonal
        and meridional spectra (of u, v or SSH) and identifying the high wavenumbers / low
        wavelengths at which the spectral energy falls precipitously.] This means the model is not
        resolving a number of submesoscale coherent vortices (SCVs) that may occur as a result of
        ML baroclinic instability; these have diameters close to 5 km. So the follow-up questions
        from a dynamical point-of-view become (1) where does the unresolved energy come from or
        go to? and (2) how does this affect your results? These questions need to be addressed or at
        least discussed if the main topic of your paper is submesoscale dynamics.}\\

        \bdp
            We are aware of the this issue --- the effective resolution based on the KE spectrum
            was already mentioned
            in the supporting information to the first draft.  In
            fact, the smallest scale discussed in our paper (10 km) was set by the effective
            resolution of highest-resolution simulation. To be more explicit, we have now included
            the effective resolution together with the nominal resolutions in the main text (Section 2).\\

            The ``unresolved energy'' does not come from anywhere --- flows smaller than
            about 10 km are strongly damped.
            To the extent that 1-km-scale flows are associated with a forward
            energy transfer, then their effects on the scales concerning our study
            (10-100 km) are likely energetically insignificant.
            Also important to this discussion is that higher resolution simulations will resolve {\it both}
            1-km-scale eddies and 1-km-scale IGWs that are generated via wave-wave
            interactions and wave-vorticity interactions.\\

            \cite{buckingham_etal2016} made a good point that models and observations
            to date may not resolve the mixed-layer deformation scale in summer; this is no
            different for our simulations. But the interpretation that this is the main
            source of seasonally varying eddy kinetic energy is unclear because
            summer is also the season when lateral buoyancy
            gradients (and thus available potential energy) are minimum within the mixed layer
            \citep[e.g., ][]{callies_etal2015}.\\

            For geostrophic mixed-layer instabilities (which we call
            shallow baroclinic instabilities), linear theory
            predict maximum growth rates at about $2.6\times R_{ml}$, where
            $R_{ml}$ is the mixed-layer deformation scale. But in equilibrated
            QG turbulence driven by ML instabilities \citep{callies_etal2016} the
            deformation scale has only a catalytic role, where eddy baroclinic
            energy is converted into (mixed-layer) eddy barotropic energy ---
            most of the baroclinic (mean-to-eddy) energy
            conversion occurs at scales much larger than the mixed-layer deformation
            scale \citep{callies_etal2016}.\\

            We have added a paragraph discussing these issues (lines 204-210).

        \edp

 \item {\it If the authors could estimate the magnitude of the sea surface height signature within a
        particular band as a function of season that might be helpful. For example, the authors could
        integrate SSH spectra (Figure 4b) from hourly fields between 10 km and 33 km (i.e.,
        wavenumber = 3 x 10-2 km-1 which appears to be the cross-over point for the two curves) to
        obtain SSH variances within a particular wavenumber band. Additionally, under the
        assumption that SSH variance in winter is dominated by balanced motions while SSH
        variance in summer is the sum of balanced + unbalanced motions, the difference between
        these two quantities yields a bound1 for the SSH variance due to unbalanced motions.}\\

        \bdp
            We have integrated the wavenumber-frequency
            spectrum of SSH (Figure 4) to estimate variance in particular frequency
            bands.  We have included a table (table 1) with the standard
            deviation of SSH for super-inertial and sub-inertial flows at scales
            smaller than 100 km in different seasons. Thanks for the suggestion!
        \edp

\end{enumerate}

\subsection{Minor comments}

\begin{enumerate}

  \item {\it (Section 1, Paragraph 1) The authors use the term ``submesoscale'' and ``mesoscale'' as
        descriptive of motions having (1-100-km) and (100-300 km) horizontal scales,
        respectively. I find these definitions misleading.

        One way to define mesoscale and submesoscale is to define what scales of motion a
        particular sensor will resolve. While this would not be optimal, it is objective. This is
        probably the motivation for the definition used by Callies and Ferrari (2013), since the
        swath altimeter (as initially proposed) would resolve scales greater than 1 km but
        smaller than the present-day, profiling satellite altimeters – i.e., eddy-like features
        with diameters greater than 80-100 km (see Chelton et al. 2011, Progress).
        Unfortunately, this definition has the adverse effect of changing for each sensor.

        A better definition of the submesoscale might be founded on dynamical arguments.
        Wunsch and Stammer (1998) introduce the geostrophic equations and note that the
        lower limit for which this approximation is strictly valid is ~30 km. This conveniently
        defines the lower limit of the mesoscale regime. I would therefore define the
        submesoscale regime as anything smaller than this but one for which the hydrostatic
        approximation remains valid (e.g., 1 km).

      The motivation for wanting to use the term ``submesoscale'' in the title is clear –
      submesoscale motions are a hot topic within the community and it catches individuals’
      eyes. This is fine. But what must be made clear within the manuscript is what scales of
      motion and what force balances (i.e., dynamical regimes) are being described.}\\

      \bdp
          We respect the reviewer's opinion about the definition of submesoscales.
          But there seems to be no clear definition in the literature --- the
          Stammer $\&$ Wunsch paper states that their
          definition of mesoscales is ``very rough'' and ``there is no generally
          agreed upon definition''. There is also irritation that oceanographers
          decided to term mesoscales the flows with dynamics analogous to
          the meteorological synoptic scale motions \citep[Appendix C of ][]{wunsch2015}.\\

          Some investigators define mesoscales as flows with scales near the
          1st baroclinic deformation radius. But there is significant confusion
          owing to factors of  $2 \pi$ --- e.g., some investigators would use $R_1 = 30$ km,
          others would argue for $L_1 = 2\pi \times 30 \approx 190$ km
          \citep{wunsch_ferrari2004,ferrari_wunsch2009}.
          To add yet another controversy, \cite{larichev_held1995} argue
          that the deformation length is not the panacea that linear theorists
          suggest. In baroclinic geostrophic turbulence, the bulk of the
          energy production occurs on scales much larger than the deformation
          radius; the energy-containing scales of equilibrated baroclinic geostrophic
          turbulence are also larger than the deformation radius \cite{larichev_held1995}.\\

          According
          to our definition the mesoscales are the energy-containing scales of the
          flow --- the eddies that contain most of the eddy kinetic energy in the ocean
          \citep{ferrari_wunsch2009}. From the the centroid of KE wavenumber
          spectrum from current altimeters, this scale varies roughly from 100-300 km.
          We then \textit{define} submesoscales as
          the energetically subdominant scales, the scales smaller than about 100 km.
          Our lower wavelength definition is limited by the effective resolution
          of our higher resolution simulation ($\sim$10 km).
          This definition is consistent with \cite{callies_ferrari2013} (they did not
          define mesoscales/submesoscales based on scales that a sensor resolves
          as the reviewer suggests).\\

          We understand that our definition is imperfect --- we may satisfy some readers but
          irritate others. But
          besides the unsettled definitions in our community and consistency with our recent
          work \citep{rocha_etal2016}, there are at least two reasons for sticking to
          our definition: (i) this is the definition used by the SWOT community (Fu \& Ferrari
              2008); incidentally current Ku-band altimeters do not resolve
              scales smaller then about 100 km; and (ii) our goal is to determine which flows
              (geostrophically balanced turbulence, inertia-gravity waves, etc)
              dominate the 10-100 km horizontal
              scales and their seasonality. In any event,  readers of the first sentence of our
              abstract will find our explicit definition of submesoscales.
      \edp


  \item {\it (Abstract) The abstract could be rewritten to emphasize its connection to SWOT.}\\

        \bdp
          We now state that there are implications for the accuracy of high-resolution
          altimeters.
        \edp

  \item {\it (Section 1, last paragraph) Adding “with implications for SWOT” at the end of the
          introduction would also help this article.}

          \bdp
            Added.
          \edp

  \item {\it (Section 1, Paragraph 1) The authors mention a suite of papers documenting
        seasonality at the submesoscale but fail missed a few: Ostrovskii (1995) and
        Brannigan et al. (2015). These are observation- and model-based studies, resp.}\\

        We now reference to the recent paper byt Brannigan et al. (2015).
        Ostrovskii (1995) can accessed via Buckingham et al. (2016).

  \item {\it (Section 1, Paragraph 2) Model resolutions: it would be helpful for readers to place in
        parentheses nominal horizontal resolutions corresponding to these spatial resolutions.}

        \bdp
          We have added both nominal and effective resolutions.
        \edp

  \item {\it (Section 1, Paragraph 3) What types of horizontal and temporal scales for IGWs are
        we talking about here? How would this fall onto the Garrett-Munk (GM) internal
        wave spectrum? (Recall: the GM spectrum contains both balanced and unbalanced
        flows, not just internal waves, despite that the name suggests this.)}\\

        \bdp
          We explicitly state that those are submesoscale IGWs. Using
          our definition of submesoscales, these are IGWs
          with horizontal scales about 10-100 km\footnote{Bill Young suggested the
          term submesoscale IGWs.}; IGWs have periods between the inertial and buoyancy
          periods.  The GM spectrum is inaccurate
          close to the surface (most of our results concern surface currents and SSH,
          and it does not account for tidal and inertial peaks. The
          new plot of the wavenumber-frequency spectrum of the SSH
          variance (new Figure 4) --- following the reviewer's suggestion ---
          should more clearly convey the idea that there is no horizontal scale
          separation between submesoscale IGWs and submesoscale turbulence.\\

          If the reviewer is using ``balanced'' as a shorthand to ``geostrophically
          and hydrostatically
          balanced'', then the comment about the GM spectrum containing both balanced
          and unbalanced flows is incorrect.  GM used linear theory
          of IGW to synthesize observations, mostly mooring data, in the IGW band
           --- they
          fitted an analytic model to (super-inertial) frequency spectra and
          different vertical modes \citep[see][and references therein]{munk1981}.
          One could argue
          that other flows project onto similar temporal scales, but those motions
          are super-inertial (e.g., stratified turbulence), not (sub-inertial)
          balanced flows.
        \edp

 \item {\it Discretization of the underlying equations of motion. How do you expect the
        combination of vertical and horizontal resolution modifies internal wave properties
        that exist in your model? Might you be attributing characteristics to these waves that
        would be different in the real ocean?}\\

        The model resolves IGWs well, particularly those waves with periods larger
        than 1h considered in this paper --- we are restricted to the output
        frequency. Higher-frequency waves that project onto lateral scales smaller
        than 10 km are potentially inaccurate but strongly damped in our simulations.



\item {\it There should be some discussion of the seasonality of the mixed layer deformation
      radius. Buckingham et al. (2016) point out that all modeling and observational studies
      to date necessarily introduce seasonally varying energy by virtue of the deformation
      radius changing with season and falling below the grid resolution in certain seasons.
      This might be worth a comment.}

      \bdp
          See major point $\#$ 3.
      \edp

\item {\it (Section 2, Paragraph 1) ``The LLC4320 simulation is an extension of the 3-month
      long output used by Rocha et al. (2016).'' I think this is good to mention, here, so
      please keep. In contrast, in (Section 5, Paragraph 2) the authors note a consistency
      between the results in the Kuroshio Extension and those in the Drake Passage. This is
      not a consistency as it is a different location, different time. The author has already
      made the reader aware of the previous study so there is no reason to cite again.}\\

        Removed.

\item {\it (Section 5, Paragraph 1) eliminate ``spectral'' after Hanning window. Also, no need for
        quotation marks.}\\

        Removed.

\item {\it (Section 5, Paragraph 1) ``the projection of these flows onto different horizontal scales
            ...'' Have you thought about computing scale-dependent vorticity, strain rate and
            divergence? This is really what you would be doing if you did a breakdown
            of the aforementioned flows into spectral space.}

            \bdp
                Following the reviewer's suggestion, we have replaced the old section
                5 with a discussion of the wavenumber-frequency spectrum of SSH variance.
                To better illustrate the projection of different flows across
                different horizontal scales, figure 5c shows the integral of the
                wavenumber-frequency spectrum over frequencies --- the result is
                very similar to calculating the wavenumber spectrum directly from
                hourly and daily-averaged fields (old figure 4b).
            \edp

\item {\it (Section 5, Paragraph 3) Remove the exclamation mark.}\\

    Removed.

\item {\it (Section 4, first paragraph) I understand the motivation for examining the Laplacian of
      sea surface height but the average reader will not understand the connection.}

      \bdp
      Readers familiar with basic geostrophy concepts are likely to understand the connection.
      We make the connection between the Laplacian of SSH and relative vorticity
      in the caption of figure 3.
      \edp

\item {\it (Section 4, second paragraph) Last sentence. This is good. I understand this but to
        make this legible for the general audience it should be explained why we expect a near
        one-to-one line for the joint-PDF of vorticity and the Laplacian of SSH.}

      As above.

\item {\it (Section 4, last paragraph) ``consistent with linear inertia-gravity waves''.
        It would be
        helpful to have a reference after this statement.}

        \bdp
        Added.
        \edp

\item {\it (Section 4) I find “jPDF” odd. Try “joint-PDF” or “PDF” in general.}\\

We changed jPDF with joint-PDF; it does look much better.

\item {\it (Section 4, last paragraph) ``...whereas divergence is moderately, negatively skewed as
        predicted by ...'' Specifically mention convergence/downwelling.}

        \bdp
            We added ``convergence/downwelling'' to the text (line xx).
        \edp

\item {\it (Figure 1) ** Place panels (d) and (e) before panels (b) and (c) since this is the order
        in which you refer to them in the manuscript. ** Difficult to see contours in (d) and
        (e). ** Consider overlaying mixed layer depth on these transects since it is mentioned
        in the text.}

        \bdp
          We have changed the order of the panels and plotted the mixed layer depth
          on the density plots.
        \edp

\item {\it (Figure 2) Is the mean of the strain rate zero? Is the mean of the divergence zero? The
        variance should be used rather than the root-mean-squared (RMS).}

        \bdp
            At every snapshot, the horizontally-averaged divergence is nearly zero;
            the strain rate horizontal average is not zero, because by definition
            $\alpha \ge 0$. The (normalized) RMS better characterizes the the bulk
            gradient Rossby number of the flow.
        \edp


\item {\it (Figure 2) What is the sensitivity of the estimates to the stencil type? See Arbic et al.
        paper describing this sensitivity for satellite altimetry.}

        \bdp
            To asses the error of centered second-order scheme we periodized
            our domain with reflections. Statistics of lateral velocity gradients
            obtained via spectral differentiation are 10-15$\%$ larger than the ones
            obtained using second-order finite differences. But there are no qualitative
            changes to Figure 2 and no changes in our conclusions. For simplicity, we
            opt to use finite differences. We now report this estimate of accuracy of
            our finite difference scheme (see text after equation 1).
        \edp

\item {\it (Section 3 and Figures 2 \& 3) At what depth are you computing these quantities?}\\

  All quantities are computed at the surface as indicated in the title of the
  section. We now explicitly indicate that in the caption of figures 2 \& 3.

\item {\it (Section 2, last paragraph) It is not just solar radiation that enters the surface buoyancy
      flux calculation.}

      We rewrote that sentence.

\item {\it (Section 2, last paragraph) ``as shallow as 40 m'' – I bet that the mixed layer depths get
      even shallower than this. At midlatitudes, 20-30 m is not uncommon.}\\

      We rewrote that sentence.

\item {\it 24. (Section 3, Paragraph 1) ``These diagnostic highlight the submesoscale structures in
      the flow.'' It might help the authors to motivate their calculations using the following:}\\

      {\it ``Any horizontal velocity field, uh, can be expressed in terms of vorticity, divergence,
      strain rate and mean flow. This can be seen by expressing uh as a Taylor series
      expansion and then decomposing the velocity gradient into symmetric and anti-
      symmetric parts [Landau and Lifshitz, 1987]. Because vortices, vertical motion and
      fronts are ubiquitous at the submesoscale, this is a natural decomposition and
      variances of these quantities are apt descriptors of submesoscale turbulence.''}\\

      \bdp
        We included a brief description/motivation for the use of second-order statistics
        of those quantities (lines 81-83). Thanks for pointing this out.
      \edp


\item {\it (Figure 3) Would be nice to include a third row that contains the collapsed (i.e., one-
      dimensional) PDFs of normalized vorticity to illustrate the skewness of hourly and
      daily-averaged flow. I believe you already have this information in the supplementary
      material but would be nice to illustrate to the reader. Also, in your normalization of
      vorticity, are you accounting for the fact that the Coriolis parameter changes with
      latitude? Your domain spans considerable changes in f and the relevant dynamical
      parameter is the gradient Rossby number.}

      \bdp
        We have experimented with a new figure 3, including a third row to show the
        vorticity PDFs. But the extra plots add little extra, if any, information
        and take a lot of space. For what it is worth, we explicitly give the relevant statistic
        (skewness) for each month, etc.  As the reviewer mentions, the collapsed
        PDFs for vorticity and divergence are shown in Figure S2, which compares
        April/October and hourly/daily-averaged in a single plot.\\


        Yes, as we mention in the text, we have normalized by those
        quantities by the \textit{local} inertial frequency, not the inertial
        frequency at mid-latitude.
      \edp

\item {\it (Figure 4b) Strictly speaking, the label on the y-axis should not be SSH variance. SSH
      variance would be the integral of this curve. The y-axis should be labelled as the
      power spectral density of sea surface height (SSH), or SSH Spectra.}\\

      \bdp
      We label the y-axis of the SSH variance spectrum as \textit{SSH variance density}
      --- short for \textit{SSH variance spectral density} --- not \textit{SSH variance} as the reviewer
      suggests.  We chose to use the short term to avoid cluttering the label
      since there is no ambiguity because we include the
      units of SSH variance per unit wavenumber.
      We further disagree that the y-axis should be labelled \textit{power spectral density of SSH}
      or \textit{SSH spectra}. On physical grounds, \textit{power} adds
      unnecessary ambiguity,
      because SSH variance density has nothing to do with \textit{power}.\footnote{The
      term \textit{power spectral density}  appears to have been introduced
      for \textit{frequency} spectrum of various quantities in signal processing (Rob Pinkel, personal
      communication).}
      And ``SSH [wavenumber] spectrum'' is the plot of SSH variance spectral density
      as a function of wavenumber, not the quantity in the y-axis.
      \edp



\item {\it The authors may find the work of Brüggeman and Eden (2015) helpful in addressing
      some of the aforementioned questions.}\\

      Thanks.


\end{enumerate}

\bibliography{rocha_etal.bib}


\end{document}
